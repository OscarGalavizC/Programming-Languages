\documentclass{article}
\usepackage[utf8]{inputenc}
\usepackage{graphicx}
\usepackage{float}
\usepackage{fancyvrb}
\title{Lenguaje PROC}
\author{Oscar Eduardo Galaviz Cuen}
\date{7 de Septiembre de 2022}

\begin{document}

\maketitle
    Este lenguaje está basado en LET pero contempla la creación e invocación de procedimientos.
    
    \section{\textbf{Sintaxis}}
    \textbf{Sintaxis concreta}\\
    $Expression ::= $ \textbf{ }$Number$\\
    $Expression ::=$ \textbf{ -}$(Expression,$ $Expression)$\\
    $Expression ::=$ \textbf{ zero?}$(Expression)$\\
    $Expression ::=$ \textbf{ if }$Expression$\textbf{ then }$Expression$\textbf{ in }$Expression$\\
    $Expression ::= $ \textbf{ }$Identifier$\\
    $Expression ::=$ \textbf{ let }$Identifier=$ $Expression$\textbf{ in }$Expression$\\
    $Expression ::=$ \textbf{ proc }$(Identifier)$\textbf{ }$Expression$\\
    $Expression ::=$ \textbf{  }$(Expression$ $Expression)$\\
    \\
    \textbf{Sintaxis abstracta}\\
    \begin{verbatim}
    (const-exp num)
    (diff-exp exp1 exp2)
    (zero?-exp exp1)
    (if-exp exp1 exp2 exp3)
    (var-exp var)
    (let-exp var exp1 body)
    (proc-exp var body)
    (call-exp op-exp arg-exp)
    \end{verbatim}
    
    \section{\textbf{Semántica}}
    Los valores expresados $ExpVal$ (valores de expresiones) y los valores denotados $DenVal$ (valores asociados en entornos) son los mismos y corresponden a $Int+Bool+Proc$. Los procedimientos expval→num,expval→bool y expval→proc toman valores expresados y regresan los valores codificados en el lenguaje de implementación.\\
    
    \textbf{Interpretación de expresiones}\\
    \begin{verbatim}
    (value-of (const-exp n) env) = (num-val n)
    
    (value-of (var-exp var) env) = env(var)
    
    (value-of (diff-exp exp1 exp2) env)
      = (num-val (- (expval->num (value-of exp1 env))
                    (expval->num (value-of exp2 env))))
                    
    (value-of (zero?-exp exp1) env)
      = (let ([val1 (value-of exp1 env)])
           (bool-val (= 0 (expval->num val1))))
           
    (value-of (if-exp exp1 exp2 exp3) env)
      = (if (expval->bool (value-of exp1 env))
            (value-of exp2 env)
            (value-of exp3 env))
    
    (value-of (let-exp var exp1 body) env)
      = (let ([val1 (value-of exp1 env)])
           (value-of body [var = val1]env))
           
    (value-of (proc-exp var body) env)
      = (proc-val (procedure var body env))
      
    (value-of (call-exp op-exp arg-exp) env)
      = (let ([proc (expval->proc (value-of op-exp env))]
              [arg (value-of arg-exp env)])
           (apply-procedure proc arg))
           
    donde:
    (apply-procedure (procedure var body env) val)
       =(value-of body [var = val]env)
    \end{verbatim}
    
    \textbf{Estrategias de implementación}\\
    Representación procedural:\\
    \begin{verbatim}
        (define (procedure var body env)
          (lambda (val)
             (value-of body [var = val]env)))
             
        (define (apply-procedure proc1 val)
          (proc1 val))
    \end{verbatim}
    Representación con estructuras de datos:\\
    \begin{verbatim}
        (define-datatype proc proc?
          (procedure
             (var identifier?)
             (body expression?)
             (saved-env environment?)))
        
        (define (apply-procedure proc1 val)
          (cases proc proc1
             (procedure (var body env)
                (value-of body [var = val]env))))
    \end{verbatim}
\end{document}
